%Dies ist die Vorlage für die Version ohne Noten

\subsection*{Freude, schöner Götterfunken}
\index{Freude, schöner Götterfunken}%

\thestrophe Freude, schöner Götterfunken, Tochter aus Elysium! \\
Wir betreten feuertrunken, Himmlische, dein Heiligtum. \\
Deine Zauber binden wieder, was die Mode streng geteilt: \\
alle Menschen werden Brüder, wo dein sanfter Flügel weilt. \\
Seid umschlungen, Millionen! Diesen Kuß der ganzen Welt! \\
Brüder, überm Sternenzelt\wdh{muß ein lieber Vater wohnen!}

\thestrophe Wem der große Wurf gelungen, eines Freundes Freund zu sein, \\
wer ein holdes Weib errungen, mische seinen Jubel ein! \\
Ja, wer auch nur eine Seele sein nennt auf dem Erdenrund! \\
Und wer's nie gekonnt, der stehle weinend sich aus diesem Bund! \\
Was den großen Ring bewohnet, huldige der Sympathie! \\
Zu den Sternen leitet sie,\wdh{wo der Unbekannte thronet.}

\thestrophe Freude trinken alle Wesen an den Büsten der Natur, \\
alle Guten, alle Bösen folgen ihrer Rosenspur. \\
Küsse gab sie uns und Reben, einen Freund, geprüft im Tod; \\
Wollust ward dem Wurm gegeben, und der Cherub steht vor Gott. \\
Ihr stürzt nieder Millionen? Ahnest du den Schöpfer, Welt? \\
Such' ihn überm Sternenzelt!\wdh{Über Sternen muß er wohnen.}

\thestrophe Freude heißt die starke Feder in der ewigen Natur. \\
Freude, Freude treibt die Räder in der großen Weltenuhr. \\
Blumen lockt sie aus den Keimen, Sonnen aus dem Firmament, \\
Sphären rollt sie in den Räumen, die des Sehers Rohr nicht kennt. \\
Froh, wie seine Sonnen fliegen durch des Himmels prächt'gen Plan, \\
laufet, Brüder, eure Bahn,\wdh{freudig wie ein Held zum Siegen!}

\thestrophe Aus der Wahrheit Feuerspiegel lächelt sie den Forscher an. \\
Zu der Tugend steilem Hügel leitet sie des Dulders Bahn. \\
Auf des Glaubens Sonnenberge sieht man ihre Fahnen wehn, \\
durch den Riß gesprengter Särge sie im Chor der Engel stehn. \\
Duldet mutig, Millionen! Duldet für die bess're Welt! \\
Droben überm Sternenzelt\wdh{wird ein großer Gott belohnen.}

\thestrophe Göttern kann man nicht vergelten, schön ist's ihnen gleich zu sein. \\
Gram und Armut soll sich melden, mit dem Frohen sich erfreun. \\
Groll und Rache sei vergessen, unserm Todfeind sei verziehn, \\
keine Träne soll ihn pressen, keine Reue nage ihn. \\
Unser Schuldbuch sei vernichtet! Ausgelöscht die ganze Welt! \\
Brüder, überm Sternenzelt\wdh{richtet Gott, wie wir gerichtet.}

\thestrophe Freude sprudelt in Pokalen, in der Traube goldnem Blut \\
trinken Sanftmut Kannibalen, die Verzweiflung Heldenmut. \\
Brüder, fliegt von euren Sitzen, wenn der volle Römer kreist, \\
laßt den Schaum zum Himmel spritzen: Dieses Glas dem guten Geist! \\
Den der Sterne Wirbel loben, den des Seraphs Hymne preist, \\
dieses Glas dem guten Geist\wdh{überm Sternenzelt dort oben!}

\thestrophe Festen Mut in schweren Leiden, Hilfe, wo die Unschuld weint, \\
Ewigkeit geschwor'nen Eiden, Wahrheit gegen Freund und Feind, \\
Männerstolz vor Königsthronen, - Brüder, gält'es Gut und Blut - \\
dem Verdienste seine Kronen, Untergang der Lügenbrut! \\
Schließt den heil'gen Zirkel dichter, schwört bei diesem goldnen Wein, \\
dem Gelübde treu zu sein, \wdh{schwört es bei dem Sternenrichter!}