%Dies ist die Vorlage für die Version ohne Noten

\subsection*{Alles schweige! Jeder neige}
\index{Alles schweige! Jeder neige}%

\weise{Landesvaterzeremonie - Vor der Burschung}

\thestrophe Alles schweige! Jeder neige ernsten Tönen nun sein Ohr! \\
\wdh{Hört, ich sing das Lied der Lieder! Hört es, meine Bundesbrüder! \\
\wdh{Hall es} wider, froher Chor!}

\thestrophe Öst'rreichs Söhne, laut ertöne euer Vaterlandsgesang! \\
\wdh{Vaterland! Du Land des Ruhmes, weih' zu deines Heiligtumes \\
\wdh{Hütern} uns und unser Schwert!}

\thestrophe Hab und Leben dir zu geben, sind wir allesamt bereit, \\
\wdh{sterben gern zu jeder Stunde, achten nicht der Todeswunde, \\
\wdh{wenn das} Vaterland gebeut.}

\thestrophe Wer's nicht fühlet, selbst nicht zielet stets nach treuer Männer \\
Wert, \wdh{soll nicht unsern Bund entehren, nicht bei diesem Schläger schwören, \\
\wdh{nicht ent-} weih'n das starke Schwert.}

\thestrophe Lied der Lieder, hall' es wider: groß und stark sei unser Mut! \\
\wdh{Seht hier den geweihten Degen, tut, wie brave Burschen pflegen, \\
\wdh{und durch-} bohrt den freien Hut!}

\thestrophe Seht ihn blinken in der Linken diesen Schläger, nie entweiht! \\
\wdh{Ich durchbohr' den Hut und schwöre, halten will ich stets auf Ehre; \\ 
\wdh{stets ein} braver Bursche sein.}

\thestrophe Nimm den Becher, wack'rer Zecher, vaterländ'schen Trankes voll; \\ 
\wdh{nimm den Schläger in die Linke, bohr ihn durch den Hut und trinke \\
\wdh{auf des} Vaterlandes Wohl!}

\thestrophe Seht ihn blinken in der Linken diesen Schläger, nie entweiht! \\
\wdh{Ich durchbohr' den Hut und schwöre, halten will ich stets auf Ehre; \\
\wdh{stets ein} braver Bursche sein.}

\thestrophe Komm, du blanker Weihedegen, freier Männer freie Wehr! \\
bringt ihn festlich mir entgegen von durchbohrten Hüten schwer.

\thestrophe Laßt uns festlich ihn entlasten; jeder Scheitel sei bedeckt, \\
und dann laßt ihn unbefleckt bis zu nächsten Feier rasten!

\thestrophe Auf, ihr Festgenossen, achtet uns're Sitte heilig schön! \\
Ganz mit Herz und Seele trachtet, stets als Männer zu besteh'n. \\
Froh zum Fest, ihr trauten Brüder; jeder sei der Väter wert! \\
Keiner taste je an's Schwert, der nicht edel ist und bieder!

\thestrophe So nimm ihn hin dein Haupt will ich bedecken und drauf den \\
Schläger strecken: es leb' auch dieser Bruder hoch! \\
Ein Hundsfott, wer ihn schimpfen sollt! So lange wir ihn kennen, \\
woll'n wir ihn Bruder nennen: es leb' auch dieser Bruder hoch!

\thestrophe Ruhe von der Burschenfeier, blanker Weihedegen, nun! \\
Jeder trachte, wackrer Freier um das Vaterland zu sein! \\
Jedem Heil, der sich bemühe, ganz zu sein der Väter wert; \\
keiner taste je ans Schwert, der nicht edel ist und bieder.