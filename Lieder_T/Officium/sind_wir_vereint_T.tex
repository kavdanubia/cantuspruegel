%Dies ist die Vorlage für die Version ohne Noten

\subsection*{Sind wir vereint zur guten Stunde}
\index{Sind wir vereint zur guten Stunde}%

\thestrophe Sind wir vereint zur guten Stunde, wir starker, froher Männerchor, \\
so dringt aus jedem frohen Munde die Seele zum Gebet hervor: \\
denn wir sind hier in ernsten Dingen mit hehrem, heiligem Gefühl; \\
\wdh{drum muß die volle Brust erklingen ein volles helles Saitenspiel.}

\thestrophe Wem soll der erste Dank erschallen? Dem Gott, der groß und wunderbar \\
aus langer Schande Nacht uns allen im Flammenglanz erschienen war; \\
der uns'rer Feinde Trotz verblitzet, der uns're Kraft uns schön erneut \\
\wdh{und auf den Sternen waltend sitzet von Ewigkeit zu Ewigkeit.}

\thestrophe Wem soll der zweite Wunsch ertönen? Des Vaterlandes Majestät! \\
Verderben allen, die es höhnen! Glück dem, der mit ihm fällt und steht! \\
Es geh', durch Tugenden bewundert, geliebt durch Redlichkeit und Recht, \\
\wdh{stolz von Jahrhundert zu Jahrhundert, an Kraft und Ehren ungeschwächt.}

\thestrophe Das dritte, freier Männer Weide, am hellsten soll's geklungen sein! \\
Die Freiheit heißet uns're Freude, die Freiheit führet uns're Reih'n; \\
für sie zu leben und zu sterben, das flammt durch jede Männerbrust; \\
\wdh{für sie um hohen Tod zu werben, ist uns're Ehre, uns're Lust.}

\thestrophe Das vierte -hebt zur hehren Weihe die Hände und die Herzen hoch! \\
Es lebe alte Männertreue, es lebe unser Glaube hoch! \\
Mit diesen wollen wir's bestehen, sie sind des Bundes Schild und Hort; \\
\wdh{fürwahr, es muß die Welt vergehen, vergeht das feste Männerwort!}

\thestrophe Rückt dichter in der heil'gen Runde und klingt den letzten Jubelklang! \\
Von Herz zu Herz, von Mund zu Munde erbrause freudig der Gesang. \\
Das Wort, das unsern Bund geschürzet, das Heil, das uns kein Teufel raubt \\
\wdh{und kein Tyrannentrug uns kürzet, das sei gehalten und geglaubt!}