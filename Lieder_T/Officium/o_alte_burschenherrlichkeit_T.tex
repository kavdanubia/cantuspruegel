%Dies ist die Vorlage für die Version ohne Noten

\subsection*{Wenn alle untreu werden}
\index{Wenn alle untreu werden}%

\comment{Vor der Philistrierung}

\thestrophe 0 alte Burschenherrlichkeit, wohin bist du entschwunden? \\
Nie kehrst du wieder, gold'ne Zeit, so froh und ungebunden! \\
Vergebens spähe ich umher, ich finde deine Spur nicht mehr. \\
\wdh{0 jerum,jerum,jerum! 0 quae mutatio rerum!}

\thestrophe Den Burschenhut bedeckt der Staub, es sank der Flaus in Trümmer, \\
der Schläger ward des Rostes Raub, verblichen ist sein Schimmer, \\
verklungen der Kommersgesang, verhallt Rapier- und Sporenklang. \\
\wdh{0 jerum,jerum,jerum! 0 quae mutatio rerum!}

\thestrophe Wo sind sie, die vom breiten Stein nicht wankten und nicht \\
wichen, die ohne Moos bei Scherz und Wein den Herrn der Erde glichen? \\
Sie zogen mit gesenktem Blick in das Philisterland zurück. \\
\wdh{0 jerum,jerum,jerum! 0 quae mutatio rerum!}

\thestrophe Da schreibt mit finsterm Angesicht der eine Relationen, \\
der andre seufzt beim Unterricht und der macht Rezensionen; \\
der schilt die sünd'ge Seele aus, und der flickt ihr verfall'nes Haus. \\
\wdh{0 jerum,jerum,jerum! 0 quae mutatio rerum!}

\thestrophe Auf öder Strecke schraubt und spannt das Fadenkreuz der \\
eine, der andre seufzt beim Blockverband, und der setzt Ziegelsteine; \\
der kocht aus Rüben Zuckersaft, und der aus Wasser Pferdekraft. \\
\wdh{0 jerum,jerum,jerum! 0 quae mutatio rerum!}

\thestrophe Zur Börse schnell der eine rennt, zu tätigem Geschäfte, \\
der andre sitzt beim Kontokorrent, und der nützt fremde Kräfte; \\
der importiert aus Turkestan, und der bohrt seine Schuldner an. \\
\wdh{0 jerum,jerum,jerum! 0 quae mutatio rerum!}

\thestrophe Allein, das rechte Burschenherz kann nimmermehr erkalten; \\
im Ernste wird, wie hier im Scherz, der rechte Sinn stets walten. \\
Die alte Schale nur ist fern, geblieben ist uns doch der Kern, \\
\wdh{und den laßt fest uns halten!}

\thestrophe Drum, Freunde, reichet euch die Hand, damit es sich erneue, \\
der alten Freundschaft heil'ges Band, das alte Band der Treue. \\
Stoßt an und hebt die Gläser hoch, die alten Burschen leben noch, \\
\wdh{noch lebt die alte Treue!}