%Dies ist die Vorlage für die Version ohne Noten

\subsection*{Es hatten drei Gesellen}
\index{Es hatten drei Gesellen}%
\index{Burschung}%

\weise{Nach der Burschung - Trauerkneipe}

\thestrophe Es hatten drei Gesellen ein fein Kollegium; \\
\wdh{es kreiste so fröhlich der Becher in dem kleinen Kreise herum.}

\thestrophe Sie lachten dazu und sangen und waren froh und frei, \\
\wdh{des Weltlaufs Elend und Sorgen, sie gingen an ihnen vorbei.}

\thestrophe Da starb von den dreien der eine, der andere folgt' ihm nach, \\
\wdh{und es blieb der dritte alleine in dem öden Jubelgemach.}

\thestrophe Und wenn die Stunde gekommen des Zechens und der Lust, \\
\wdh{dann täl er die Becher füllen und sang aus voller Brust.}

\thestrophe So saß er einst auch beim Mahle und sang zum Saitenspiel \\
\wdh{und zu dem Wein im Pokale eine helle Träne fiel.:}

\thestrophe "Ich trink' euch ein Schmollis, ihr Brüder! \\
Wie sitzt ihr so stumm und so still? \\
\wdh{Was soll aus der Welt denn noch werden, wenn keiner mehr trinken will?"}

\thestrophe Da klangen der Gläser dreie und wurden mählich leer: \\
\wdh{"Fiducit, fröhlicher Bruder!" -Der trank keinen Tropfen mehr.}