%Dies ist die Vorlage für die Version ohne Noten

\subsection*{Sind wir vereint zur guten Stunde - Neufassung}
\index{Sind wir vereint zur guten Stunde - Neufassung}%

\comment{Helmar Kögl v/o Dr.cer. Arno, Dan! 2006}

\thestrophe Sind wir vereint zur guten Stunde, wir fröhliche Studentenschaft, \\
gemeinsam in der hehren Runde genießen wir den Gerstensaft. \\
Denn wir sind hier ein Fest zu feiern mit Freunden und mit Bierverstand, \\
\wdh{drum soll dies Lied uns einbegleiten, dass Freundschaft immer uns verband.}

\thestrophe Sind wir vereint zur guten Stunde, wir muntere Studentenschaft, \\
gemeinsam in der frohen Runde bewundern wir des Comments Macht. \\
Denn wir befolgen strenge Regeln mit Liedersingen und Colloquium, \\
\wdh{damit kein Bruder haut daneben und schädigt unser Eigentum.}

\thestrophe Sind wir vereint zur guten Stunde, wir christliche Studentenschaft, \\
gemeinsam in der hehren Runde bestaunen wir des Seniors Pracht. \\
Denn er ist hier das Fest zu leiten, und zu erlauben, wer da spricht, \\
\wdh{zu dirigiern wie einst in Gründungszeiten bis, dass erlöscht der Kneipe Licht.}

\thestrophe Sind wir vereint zur guten Stunde, wir feiernde Studentenschaft, \\
gemeinsam in der hehren Runde verbringen wir die lange Nacht. \\
Die Füchse sind schon im ermatten, die Burschen müd' und gleich k.o., \\
\wdh{Sie wirken fast wie bleiche Schatten, das Fest ist aus mit Jubilo.}

\thestrophe Sind wir vereint zur guten Stunde, wir zechende Studentenschaft, \\
Wir wandern heim allein vom Bunde, bis es zu Hause wieder kracht. \\
Das Schlüsselloch ist kaum zu finden, Nachbarn keifen munter los, \\
\wdh{und holdes Weib sieht man entschwinden mit bösem Blick, der gnadenlos.}