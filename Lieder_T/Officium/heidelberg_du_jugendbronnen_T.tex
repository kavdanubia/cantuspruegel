%Dies ist die Vorlage für die Version ohne Noten

\subsection*{Heidelberg, du Jugendbronnen}
\index{Heidelberg, du Jugendbronnen}%

\thestrophe Heidelberg, du Jugendbronnen, Zauberin am Neckarstrand, \\
solchen Fleck, uns warm zu sonnen, gab der Herrgott keinem Land! \\
Schläger schwirren, Gläser klingen, alles atmet Frohnatur, \\
\wdh{selbst im Laub die Vöglein singen: Gaudeamus igitur!}

\thestrophe Wohl die alte Burg voll Narben trauert um vergang'ne Zeit, \\
doch sie tut's in lichten Farben fröhlich-feuchter Traurigkeit. \\
Schaut sie so aufs viele Bürsten wie mit sanfter Rührung hin, \\
\wdh{denkt sie ihrer alten Fürsten, die so groß und stark darin.}

\thestrophe Schäumend tosten hier die Becher, und Herrn Otto Heinrich \\
galt's, der berühmter noch als Zecher, denn als Graf der schönen Pfalz. \\
Nur ein Burgzwerg traf's noch besser, der ging recte gleich zum Spund, \\
\wdh{und das größte aller Fässer schlürft' er aus bis auf den Grund!}

\thestrophe Seine Tat, so kühn gelungen, lebt im Lled unsterblich fort, \\
und der Sänger, der's gesungen, ragt in Erz gegossen dort. \\
Schar um Schar zum Scheffelhaine wogt empor auf Waldespfad, \\
\wdh{und "Altheidelberg, du Feine" summt's dort oben früh und spät!}

\thestrophe Frohe Stadt, zum Unterpfande, daß dein Glück dich nie \\
verläßt, grüßt uns hoch von Dachesrande ein verweg'nes Storchennest! \\
Ei, wie han's die lebensfrischen Weiblein hier sich gut bestellt: \\
\wdh{geht der Storch im Neckar fischen, kommt was Lustiges zur Welt!}

\thestrophe So gedeih bei Storch und Kater, fröhliche Studentenschaft! \\
Brausend kling' dein Landesvater stets bei Wein und Gerstensaft! \\
Prosit deinem Sangesmeister, Prosit deinem großen Zwerg, \\
\wdh{Scheffels und Perkeos Geister walten über Heidelberg!}