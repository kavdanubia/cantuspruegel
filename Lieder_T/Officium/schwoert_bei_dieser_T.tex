%Dies ist die Vorlage für die Version ohne Noten

\subsection*{Schwört bei dieser blanken Wehre}
\index{Schwört bei dieser blanken Wehre}%

\comment{Vor der Ehrenbandverleihung}

\thestrophe Schwört bei dieser blanken Wehre, schwört, Ihr Brüder, \\
allzumal: Fleckenrein sei unsre Ehre wie ein Schild von lichtem \\
Stahl. Was wir schwuren, sei gehalten treulich bis zur letzten Ruh'. \\
\wdh{Hört's, ihr Jungen, hört's, ihr Alten, Gott im Himmel, hör's auch du!}

\thestrophe Freiheit, duft'ge Himmelsblume, Morgenstern nach banger \\
Nacht ! Treu vor deinem Heiligtume steh'n wir alle auf der Wacht. \\
Was erstritten uns're Ahnen, halten wir in starker Hut; \\
\wdh{Freiheit schreibt auf eure Fahnen, für die Freiheit unser Blut!}

\thestrophe Vaterland, du Land der Ehre, stolze Braut mit freier Stirn! \\
Deinen Fuß benetzen Meere, deinen Scheitel krönt der Firn, \\
laß um deine Huld uns werben, schirmen dich von uns'rer Hand, \\
\wdh{dein im Leben, dein im Sterben, ruhmbekränztes Vaterland!}

\thestrophe Schwenkt der Schläger blanke Klingen, hebt die Becher, \\
stoßet an! Unser Streben, unser Ringen, aller Welt sei's kundgetan! \\
Laßt das Burschenbanner wallen, haltet's hoch mit starker Hand, \\
\wdh{brausend laßt den Ruf erschallen: Ehre, Freiheit, Vaterland!}