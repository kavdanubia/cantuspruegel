%Dies ist die Vorlage für die Version ohne Noten

\subsection*{Wien, mein Wien, gar oft besungen}
\index{Wien, mein Wien, gar oft besungen}%
\index{Studienstadt@Wien}%
%
\nopagebreak
\hfill%
\comment{Studienstadt: Wien}
\weise{Heidelberg, du Jugendbronnen}
%
\nopagebreak
\thestrophe Wien, mein Wien, gar oft besungen, an der blauen Donau Strand, \\
schon im Lied der Nibelungen bist ein Kleinod du genannt. \\
Wien, mein Wien, du Stolz der Ahnen, schönes Wien, demantengleich, \\
\wdh{fröhlich flattern deine Fahnen in dem schönen Österreich!}

\thestrophe Wo dem Nebeldunst enthoben dich der Wienerwald umrauscht,\\
dort die alte Feste droben mit dem Himmel Zwiesprach' tauscht. \\
Drunten aber fest und trutzig, treu bewährt in manchem Sturm, \\
\wdh{grüßt uns aus dem Häusermeere unser alter Stephansturm!}

\thestrophe Prinz Eugen und Kaiser Josef, Mozart, Raimund, Schubert Franz, \\
sind im Herzen uns verwoben wie der alte Wiener Tanz. \\
Alte, liebe Wiener Sitte, alte Wiener Fröhlichkeit, \\
\wdh{Strauß und Lanner, Wiener Walzer, leben fort in Ewigkeit!}

\thestrophe Und da draußen vor den Toren grüßt ein Städtchen traut im Land, \\
wo das Liebste uns geboren, das auf Erden uns bekannt. \\
Wo das blau-weiß-gold'ne Banner flatterte zuerst im Wind, \\
\wdh{in Korneuburgs engen Gassen fühlt man als der Heimat Kind!}

\thestrophe So bleibt ihr uns stets verwoben: Stadt und du, mein Städtchen traut, \\
euch zum Preise sei erhoben, was Gambrinus uns gebraut. \\
Ob uns mahnen Kirchenglocken, ob uns ruft Kommersgesang, \\
\wdh{ob zum Tanz die Geigen locken: stets ist's unsrer Heimat Klang!}
