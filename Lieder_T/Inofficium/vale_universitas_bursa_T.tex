%Dies ist die Vorlage für die Version ohne Noten

\subsection*{Vale universitas}
\index{Vale universitas}%

\thestrophe Vale universitas, Bursa und Taberne! Blumen dringen durch \\
das Gras und uns lockt die Ferne. Zwar faßt unser fahrend' Gut leicht \\
ein winzig Tüchlein, doch was schad't's? Was not uns tut, \\
schafft das Zaubersprüchlein: Sumus de vagantium ordine laudando, \\
petimus viaticum porro properando, porro properando.

\thestrophe Abbas illustrissimus ist in jungen Jahren auch als vagans \\
clericus durch das Land gefahren. Drum winkt er dem Kellner \\
gleich, hört er drauß' uns pochen, denkt der Zeiten sälderreich, \\
da er selbst gesprochen: Sumus de vagantium ordine laudando, \\
petimus viaticum porro properando, porro properando.

\thestrophe Seh'n wir im Vorrübergeh'n eine Maid im Gärtlein zwischen \\
Gilg und Rosen steh'n, klopfen wir ans Pförtlein. Neigt sie sich \\
verschämt uns zu, fragend, was wir gehren: Einen Kuß, \\
Blauäuglein Du! Einen Kuß in Ehren! Sumus de vagantium \\
ordine laudando, petimus viaticum porro properando, porro properando.

\thestrophe Vor dem Pfarrhaus schreckt ein Drach oft uns arme Pilger: \\
"Hebt euch weg, Vagantenpack! Schnöde Weinvertilger!" Doch es \\
winkt des Pfarrherrn Hand hinterm Drachen milde - das Barett \\
zieh'n wir galant vor der bösen Hilde: Sumus de vagantium ordine \\
laudando, petimus viaticum porro properando, porro properando.

\thestrophe Tat ein Schloß auch nie sich auf Feinden, die's berannten, \\
stürmen wir's im Siegeslauf, fröhliche Vaganten. Eine Tageweise \\
hell bläst zum Gruß der Türmer; Herr und Troß ergibt sich \\
schnell, schallt der Ruf der St ürmer: Sumus de vagantium ordine \\
laudando, petimus viaticum porro properando, porro properando.

\thestrophe Und wenn ab das Glück sich kehrt, uns're Wangen blassen, \\
der die jungen Raben nährt, wird uns nicht verlassen. Steht sein Bild \\
am Straßenrand, traut im Tannenreise, grüßen wir's mit Mund \\
und Hand, und dann fleh'n wir leise: Sumus de vagantium ordine \\
laudando, petimus viaticum porro properando, porro properando.