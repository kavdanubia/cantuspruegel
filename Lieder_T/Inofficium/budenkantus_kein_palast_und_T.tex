%Dies ist die Vorlage für die Version ohne Noten

\subsection*{Budenkantus}
\index{Budenkantus}%

\weise{Strömt herbei, ihr Völkerscharen}

\thestrophe Kein Palast und keine Schenke zieht mich so gewaltig hin als \\
wie die Danubenbude in der Rathausstraß' zu Wien. \\
Weil es nirgends auf der Erde schöner, froher, besser war als hier unten \\
tief im Keller an der lieben trauten Bar.

\thestrophe Wo viel Nachtgetier einst hauste, wo einst herrschte tiefste \\\\
Nacht, jubeln heute frohe Chöre, wird gesungen und gelacht. \\
Denn kein Flecken auf der Erde ist so lieb und teuer mir wie \\
Danubias hehre Bude hier bei edlem würz'gem Bier.

\thestrophe Und man glaubt, die Stiege zitt're von des Monds und Süffls \\
Schritt und es dröhnen die Gewölbe, klingen Glas und Krüge mit. \\
Sind die Gelder dann versoffen, stehen leer die Gläser nur, doch \\
noch immer hallt und schallt es: Gaudeamus igitur!

\thestrophe Mancher Rausch und mancher Affe wird im Kater hier \\
verbüßt, wenn der allzufrühe Morgen den Danuben keck begrüßt. \\
Denn zum Trinken auf der Bude ist ja jede Stunde recht, ist \\
moralisch dann im Kater uns auch hundeelend schlecht.