%Dies ist die Vorlage für die Version ohne Noten

\subsection*{Wütend wälzt sich einst im Bette}
\index{Wütend wälzt sich einst im Bette}%

\thestrophe Wütend wälzt' sich einst im Bette Kurfürst Friedrich \\
von der Pfalz, gegen alle Etikette brüllte er aus vollem Hals: \\
\wdh{"Wie kam gestern ich ins Nest? Bin, scheint's, wieder voll gewest."}

\thestrophe "Na, ein wenig schief geladen ", grinste drauf der Kammermohr, \\
"selbst von Mainz des Bischofs Gnaden kamen mir benebelt vor, \\
\wdh{'s war halt doch ein schönes Fest, alles wieder voll gewest."}

\thestrophe "So, das findest du zum Lachen, Sklavenseele, lache nur, \\
künftig will ich's anders machen, Hassan, höre meinen Schwur: \\
\wdh{'s letzte Mal bei Tod und Pest, daß ich wieder voll gewest!"}

\thestrophe "Will ein christlich Leben führen, ganz mich der Beschauung \\
weih'n, um mein Tun zu kontrollieren, trag' ich's in mein Tagbuch ein, \\
\wdh{und ich hoff', daß nie ihr lest, daß ich wieder voll gewest."}

\thestrophe Als der Kurfürst kam zu sterben, machte er sein Testament, \\
und es fanden seine Erben auch ein Buch aus Pergament. \\
Drinnen stand auf jeder Seit': "Seid vernünftig, liebe Leut', \\
\wdh{dieses geb' ich zu Attest: Heute wieder voll gewest."}

\thestrophe Hieraus mag nun jeder sehen, was ein guter Vorsatz nützt, \\
und wozu auch widerstehen, wenn der volle Becher blitzt? \\
\wdh{Drum stoßt an! Probatum est: Heute wieder voll gewest!}