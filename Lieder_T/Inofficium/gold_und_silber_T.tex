%Dies ist die Vorlage für die Version ohne Noten

\subsection*{Gold und Silber}
\index{Gold und Silber}%

\thestrophe Gold und Silber lieb' ich sehr, kann's auch gut gebrauchen, \\
hätt' ich nur ein ganzes Meer, mich hinein zu tauchen; \\
's braucht ja nicht geprägt zu sein, hab's auch so ganz gerne \\
\wdh{sei's des Mondes Silberschein, sei's das Gold der Sterne.}

\thestrophe Doch viel schöner ist das Gold, das vom Lockenköpfchen \\
meines Liebchens niederrollt in zwei blonden Zöpfchen. \\
Darum du, mein liebes Kind, laß dich herzen, küssen \\
\wdh{eh' die Locken silbern sind und wir scheiden müssen.}

\thestrophe Holdes Liebchen, trag' nicht Leid, blicke nicht so trübe, \\
weil du nicht die einz'ge Maid, die ich herzlich liebe! \\
Schau' Studenten machen's so, lieben mehr als eine, \\
\wdh{bin ich nicht mehr Studio, lieb' ich dich alleine.}

\thestrophe Gräm' dich nicht den ganzen Tag, daß wir gerne trinken, \\
daß ich dich nicht küssen mag, wenn die Becher winken. \\
Schau', Studenten sind halt so, lieben Bier und Weine, \\
\wdh{bin ich nicht mehr Studio, lieb' ich dich alleine.}

\thestrophe Wer nur eine einz'ge küßt bis zur Jahreswende, \\
und die andern schüchtern grüßt, der ist kein Studente. \\
Wer noch nie bekneipet war, der hat nie studieret, \\
\wdh{wär' er auch so manches Jahr ins Kolleg marschieret.}

\thestrophe Seht, 'wie blinkt der gold'ne Wein hier in meinem Becher; \\
horcht, wie klingt so silberrein froher Sang der Zecher; \\
daß die Zeit einst golden war, will ich nicht bestreiten, \\
\wdh{denk' ich doch im Silberhaar gern vergang'ner Zeiten.}