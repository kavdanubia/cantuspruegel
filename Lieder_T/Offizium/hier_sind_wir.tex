%Dies ist die Vorlage für die Version ohne Noten

\subsection*{Hier sind wir versammelt}
\index{Hier sind wir versammelt}%
\index{Begrüßung}%

\thestrophe Hier sind wir versammelt zu löblichem Tun, drum, Brüderchen, ergo bibamus! \\
Die Gläser, sie klingen, Gespräche, sie ruh'n, beherziget ergo bibamus! \\
Das heißt noch ein altes, ein tüchtiges Wort, es passet zum ersten und passet so fort, \\
und schallet ein Echo vom festlichen Ort, \wdh{ein herrliches ergo bibamus!}

\thestrophe Ich hatte mein freundliches Liebchen gesehn, da dacht' ich mir: ergo bibamus! \\
Und nahte mich traulich, da ließ sie mich steh'n, ich half mir und dachte: bibamus! \\
Und wenn sie versöhnet euch herzet und küßt, und wenn ihr das Herzen und Küssen \\
vermißt, so bleibet nur, bis ihr was Besseres wißt, \wdh{beim tröstlichen ergo bibamus!}

\thestrophe Mich ruft das Geschick von den Freunden hinweg: ihr Redlichen! ergo bibamus! \\
Ich scheide von hinnen mit leichtem Gepäck, drum doppeltes ergo bibamus! \\
Und was auch der Filz von dem Leibe sich schmorgt, so bleibt für den Heit'ren doch \\
immer gesorgt, weil immer dem Frohen der Fröhliche borgt: \wdh{drum Brüderchen, ergo bibamus!}

\thestrophe Was sollen wir sagen vom heutigen Tag? Ich dachte nur: ergo bibamus! \\
Er ist nun einmal von besonderem Schlag , drum immer aufs Neue: bibamus! \\
Er führet die Freude durch's offene Tor, es glänzen die Wolken, es teilt sich der Flor, \\
da leuchtet ein Bildchen, ein göttliches, vor; \wdh{wir klingen und singen: bibamus!}