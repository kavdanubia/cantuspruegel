%Dies ist die Vorlage für die Version ohne Noten

\subsection*{Dort, wo des Wienerwaldes liebes Rauschen}
\index{Dort, wo des Wienerwaldes liebes Rauschen}%
\index{Studienstadt@Wien}%

\hfill%
Studienstadt: Wien\\
\weise{?????}

\thestrophe Dort, wo des Wienerwaldes liebes Rauschen Sankt Stephans Münster \\
brudernah' begrüßt, die Donauwelle frohem Sang mag lauschen, \\
eh' scheidend sie der Ungarn Ufer küsst, \wdh{\wdh{dort zieht's mich hin,} \\
zu dir, mein teures Wien, der Städte Glanz und Königin!}

\thestrophe Du stolzes Wien, von grünem Wald umschlossen, \\
an deinen Hängen wächst ein edler Wein, wer den inbrünstig einmal nur genossen, \\
der buhlt nicht mehr um Rebendank vom Rhein. \wdh{\wdh{Du Wein so hold,} \\
wie eitel flüssig Gold, bist Himmelstrank, von Gott gewollt!}

\thestrophe Und aus den Fenstern kluge Mädchen schauen mit Äuglein sinnig, \\
mild und sonnenklar, ein zücht'ger Sinn und Geist wohnt in den Frauen, \\
ein gold'nes Herz, in Liebe wunderbar. \wdh{\wdh{Aus euren Reih'n,} \\
ihr Wiener Mägdelein, will ich mein Glück, mein Weibchen frei'n!}

\thestrophe In deiner Alma mater heil'gen Räumen hab' ich nach Wissen \\
ehrenvoll gestrebt, um sie herum in Minne, Durst und Träumen \\
die herrlich flotte Burschenzeit verlebt. \wdh{\wdh{O Seligkeit} \\
der liederfrohen Zeit, wie bist du nun so weit, so weit!}

\thestrophe Drum bleibst du meinem Herzen stets verwoben, \\
das jung und keck in dir geschlagen hat; dir sei zum Preis mein volles Glas erhoben, \\
dir, unsrer Heimat schmucker Liederstadt. \wdh{\wdh{Und muss es sein,} \\
holt mich Gevatter Hein, will ich in dir begraben sein!}
