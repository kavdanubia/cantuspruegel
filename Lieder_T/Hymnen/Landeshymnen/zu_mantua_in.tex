%Dies ist die Vorlage für die Version ohne Noten

\subsection*{Zu Mantua in Banden}
\index{Zu Mantua in Banden}%
\index{Tiroler Landeshymne}%

\thestrophe Zu Mantua in Banden der treue Hof er war, \\
in Mantua zum Tode führt ihn der Feinde Schar. \\
Es blutete des Brüder Herz, ganz Deutschland, \\
ach, in Schmach und Schmerz, \wdh{mit ihm das Land Tirol. 4x}

\thestrophe Die Hände auf dem Rücken der Sandwirt \\
Hofer ging mit ruhigen festen Schritten, \\
ihm schien der Tod gering, der Tod den er so manches Mal \\
vom Iselberg geschickt ins Tal, \wdh{im heil'gen Land Tirol.} 

\thestrophe Doch als aus Kerkergittern im festen Mantua, \\
die treuen Waffenbrüder die Händ' er strecken sah, \\
da rief er laut: "Gott sei mit euch, mit dem verrat'nen deutschen Reich \\
\wdh{und mit dem Land Tirol!"}

\thestrophe Dem Tambour will der Wirbel nicht unterm Schlegel vor, \\
als nun der Sandwirt Hofer schritt durch das finst're Tor. \\
Der Sandwirt, noch in Banden frei, dort stand er fest auf der Bastei, \\
\wdh{der Mann vom Land Tirol.}

\thestrophe Dort soll er niederknien. Er sprach: "Das tu ich nit! \\
Will sterben wie ich stehe, will sterben, wie ich stritt, \\
so wie ich steh auf dieser Schanz. Es leb mein guter Kaiser Franz, \\
\wdh{mit ihm das Land Tirol!"}

\thestrophe Und von der Hand die Binde nimmt ihm der Korporal, \\
und Sandwirt Hofer betet all hier zum letzten Mal. \\
Dann ruft er: "Nun, so trefft mich recht. \\
Gebt Feuer! - Ach, wie schießt ihr schlecht! \wdh{Ade, mein Land Tirol!"}