%Oberste Ebene, Buch
\documentclass[b5paper, 11pt]{book}

\usepackage{fullpage}
\usepackage[utf8]{inputenc}
\usepackage[naustrian]{babel}
\usepackage{ifthen}
\usepackage{makeidx}
\usepackage{geometry}

\setlength{\parindent}{0pt}

\newcommand{\wdh}[1]{$|:$~#1~$:|$}
\newcommand{\mychapter}[1]{%
	\stepcounter{chapter}%
	\setcounter{section}{0}%
	\chapter*{#1}%
	\addcontentsline{toc}{chapter}{#1}%
	}
\newcommand{\comment}[1]{\hfill #1 \newline}
\newcommand{\kursiv}[1]{\textit #1}
\newcommand{\weise}[1]{\hfill Weise: #1 \newline\newline}

\newcounter{strophe}[subsection]
\renewcommand{\thestrophe}{\stepcounter{strophe}\arabic{strophe}.~}

\makeindex

\usepackage{graphics}
\begin{document}

\tableofcontents


\mychapter{Hymnen}

\input{Lieder_N/heil_danubia_lass_N.tex}
\input{Lieder_N/auf_des_glaubens_N.tex} %"auf_des_glaubens_N.lytex" wird eingebunden
%Dies ist die Vorlage für die Version ohne Noten

\subsection*{Einer Farbe, einem Glauben}
\index{Einer Farbe, einem Glauben}%
\index{MKV Hymne}%
%
\nopagebreak
\weise{Strömt herbei ihr Völkerscharen}
%
\nopagebreak
\thestrophe Einer Farbe, einem Glauben, einer Sitte zugetan, \\
häng' ich wie die frommen Tauben meiner lieben Heimat an. \\
Wo ich lebe, will ich sterben; wo ich sterbe, ruht sich's gut; \\
\wdh{und die Kinder, die mir erben, erben auch mein Herz, mein Blut.}

\thestrophe Süße Heimat, schöne Erde, gutes Land, das mich erhält, \\
o du teure, liebe, werte, o du kleine heit're Welt! \\
Immer will ich dir gehören, immer mit und bei dir sein! \\
\wdh{Fremdlinge und Söldner schwören, dir genügt mein Wort allein.}

\thestrophe Meinem Glauben, meiner Sitte, meinem Vaterlande treu, \\
kenn' ich weder Wunsch noch Bitte, frage nicht, wo's besser sei. \\
Mögen and're wünschen, suchen, mir sind über Gut und Geld \\
\wdh{meine Eichen, meine Buchen, MKV, du meine Welt!}

\input{Lieder_N/in_dem_staedtchen_N.tex}

\mychapter{Landeshymnen}

%Dies ist die Vorlage für die Version ohne Noten

\subsection*{Dort, wo Tirol an Salzburg grenzt}
\index{Dort, wo Tirol an Salzburg grenzt}%
\index{Kärntner Landeshymne}%

\thestrophe Dort, wo Tirol an Salzburg grenzt, des Glockners Eisgefilde glänzt, \\
wo aus dem Kranz, der es umschließt, der Leiter reine Quelle fließt, \\
\wdh{laut tosend, längs der Berge Rand, beginnt mein teures Heimatland.}

\thestrophe Wo durch der Matten herrlich Grün des Draustroms rasche Fluten ziehn; \\
vom Eisenhut, wo schneebedeckt sich Nordgau's Alpenkette streckt, \\
\wdh{bis zur Karawanken Felsenwand dehnt sichmein freundlich Heimatland.}

\thestrophe Wo von der Alpenluft umweht, Pomonens schönster Tempel steht, \\
wo durch die Ufer, reich umblüht, der Lavant Welle rauschend zieht, \\
\wdh{im grünen Kleid ein Silberband, schließt sich mein liebes Heimatland.}
\input{Lieder_T/Hymnen/Landeshymnen/du_laendle_meine_T.tex}
%Dies ist die Vorlage für die Version ohne Noten

\subsection*{Hoamatland}
\index{Hoamatland}%
\index{Oberösterreichische Landeshymne}%

\thestrophe Hoamatland, Hoamatland, die han i so gern, \\
\wdh{wiar a Kinderl sein Muader, a Hünderl sein Herrn.}

\thestrophe Duri 's Tal bin i g'laffn, af'n Höcherl bin i glegn, \\
\wdh{und deinSunn hat mi trückat, wann mi gnetzt hat dein Regn.}

\thestrophe Deine Bam, deine Staudna san groß wom mit mir, \\
\wdh{und sie bliahn sehen und tragn und sagn: Machts a wia mir.}

\thestrophe Am schönem macht's 's Bacherl, laft allweil tal-a, \\
\wdh{aber 's Herz, vo wo's auarinnt, 's Herz, des laßt's da.}

\thestrophe Und i und die Bachquelln san Vetter und Moahm; \\
\wdh{treibt's mi, wo da wöll, umma, mein Herz is dahoam.}

\thestrophe Dahoam is dahoam, wannst net furt muaßt, so bleib, \\
\wdh{denn d'Hoamat is ehnta da zweit Muaderleib.}


\mychapter{Orte}

\input{Lieder_N/als_ich_zog_N.tex}
\input{Lieder_T/Orte/wien_mein_wien_T.tex}

\mychapter{Officium}

\input{Lieder_T/Officium/gaudeamus_igitur_iuvenes_T.tex}
%Dies ist die Vorlage für die Version ohne Noten

\subsection*{Alles schweige! Jeder neige}
\index{Alles schweige! Jeder neige}%

\weise{Landesvaterzeremonie - Vor der Burschung}

\thestrophe Alles schweige! Jeder neige ernsten Tönen nun sein Ohr! \\
\wdh{Hört, ich sing das Lied der Lieder! Hört es, meine Bundesbrüder! \\
\wdh{Hall es} wider, froher Chor!}

\thestrophe Öst'rreichs Söhne, laut ertöne euer Vaterlandsgesang! \\
\wdh{Vaterland! Du Land des Ruhmes, weih' zu deines Heiligtumes \\
\wdh{Hütern} uns und unser Schwert!}

\thestrophe Hab und Leben dir zu geben, sind wir allesamt bereit, \\
\wdh{sterben gern zu jeder Stunde, achten nicht der Todeswunde, \\
\wdh{wenn das} Vaterland gebeut.}

\thestrophe Wer's nicht fühlet, selbst nicht zielet stets nach treuer Männer \\
Wert, \wdh{soll nicht unsern Bund entehren, nicht bei diesem Schläger schwören, \\
\wdh{nicht ent-} weih'n das starke Schwert.}

\thestrophe Lied der Lieder, hall' es wider: groß und stark sei unser Mut! \\
\wdh{Seht hier den geweihten Degen, tut, wie brave Burschen pflegen, \\
\wdh{und durch-} bohrt den freien Hut!}

\thestrophe Seht ihn blinken in der Linken diesen Schläger, nie entweiht! \\
\wdh{Ich durchbohr' den Hut und schwöre, halten will ich stets auf Ehre; \\ 
\wdh{stets ein} braver Bursche sein.}

\thestrophe Nimm den Becher, wack'rer Zecher, vaterländ'schen Trankes voll; \\ 
\wdh{nimm den Schläger in die Linke, bohr ihn durch den Hut und trinke \\
\wdh{auf des} Vaterlandes Wohl!}

\thestrophe Seht ihn blinken in der Linken diesen Schläger, nie entweiht! \\
\wdh{Ich durchbohr' den Hut und schwöre, halten will ich stets auf Ehre; \\
\wdh{stets ein} braver Bursche sein.}

\thestrophe Komm, du blanker Weihedegen, freier Männer freie Wehr! \\
bringt ihn festlich mir entgegen von durchbohrten Hüten schwer.

\thestrophe Laßt uns festlich ihn entlasten; jeder Scheitel sei bedeckt, \\
und dann laßt ihn unbefleckt bis zu nächsten Feier rasten!

\thestrophe Auf, ihr Festgenossen, achtet uns're Sitte heilig schön! \\
Ganz mit Herz und Seele trachtet, stets als Männer zu besteh'n. \\
Froh zum Fest, ihr trauten Brüder; jeder sei der Väter wert! \\
Keiner taste je an's Schwert, der nicht edel ist und bieder!

\thestrophe So nimm ihn hin dein Haupt will ich bedecken und drauf den \\
Schläger strecken: es leb' auch dieser Bruder hoch! \\
Ein Hundsfott, wer ihn schimpfen sollt! So lange wir ihn kennen, \\
woll'n wir ihn Bruder nennen: es leb' auch dieser Bruder hoch!

\thestrophe Ruhe von der Burschenfeier, blanker Weihedegen, nun! \\
Jeder trachte, wackrer Freier um das Vaterland zu sein! \\
Jedem Heil, der sich bemühe, ganz zu sein der Väter wert; \\
keiner taste je ans Schwert, der nicht edel ist und bieder.

\pagebreak
\printindex

\end{document}
