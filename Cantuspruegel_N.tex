%Oberste Ebene, Buch
\documentclass[b5paper, 11pt]{book}

\usepackage{fullpage}
\usepackage[utf8]{inputenc}
\usepackage[naustrian]{babel}
\usepackage{ifthen}
\usepackage{makeidx}
\usepackage{geometry}

\setlength{\parindent}{0pt}

\newcommand{\wdh}[1]{$|:$~#1~$:|$}
\newcommand{\mychapter}[1]{%
	\stepcounter{chapter}%
	\setcounter{section}{0}%
	\chapter*{#1}%
	\addcontentsline{toc}{chapter}{#1}%
	}
\newcommand{\comment}[1]{\hfill #1 \newline}
\newcommand{\kursiv}[1]{\textit #1}
\newcommand{\weise}[1]{\hfill Weise: #1 \newline\newline}

\newcounter{strophe}[subsection]
\renewcommand{\thestrophe}{\stepcounter{strophe}\arabic{strophe}.~}

\makeindex

\usepackage{graphics}
\begin{document}

\tableofcontents


\mychapter{Hymnen}

%Dies ist die Vorlage für die Version ohne Noten

\subsection*{Danubias Bundeslied}
\index{Danubias Bundeslied}%
\index{Bundeslied}%
\index{Heil Danubia}%
\index{Danubia@Bundeslied}
%
\nopagebreak
\weise{Schwört bei dieser blanken Wehre}
%
\nopagebreak
\thestrophe Heil, Danubia, laß es fliegen, deines Bundes Ehr und Zier. \\
In dem Kampf sei, willst du siegen, blau-weiß-golden dein Panier! \\
Auf, ihr Brüder, laßt uns schwören, reichet euch die treue Hand! \\
\wdh{Unsern Feldruf soll man hören stolz und kühn im Vaterland!}

\thestrophe Blau wie des Danubius Wogen, der aus weißem Felsen schäumt, \\
hoch, dem Schmutz der Welt entzogen, von der Sonne Gold umsäumt; \\
Blau-Weiß-Gold, führ' uns zum Siege, sei im Streite unser Schild, \\
\wdh{wenn es gegen Trug und Lüge harten Kampf zu kämpfen gilt.}

\thestrophe Dir, du blau-weiß-goldne Fahne, sei mein ganzes Herz \\
geweiht, bis ich auf des Charons Kahne fahre in die Ewigkeit! \\
Will als treuer Mann versterben, treu gen Gott und Vaterland, \\
\wdh{treu im Glück und im Verderben dir, du blau-weiß-goldnes Band.}

%In diesem Fall sind die Noten schon im Lied "strömt herbei ihr Völkerscharen.
%Daher ist es nicht notwendig Noten einzubinden

\subsection*{Auf des Glaubens Felsengrunde}
\index{Auf des Glaubens Felsengrunde}%
\index{ÖCV Hymne}%
\nopagebreak
\hfill%
Weise: Strömt herbei ihr Völkerscharen%
\nopagebreak
\thestrophe Auf des Glaubens Felsengrunde stehe du, Cartellverband, \\
wohlgeeint zu jeder Stunde, treu zu Gott und Vaterland! \\
Unserm Österreich zur Ehre, was auch bringen mag die Zeit, \\
\wdh{und zum Schutze der Altäre sieh uns, Herr, im Kampf bereit!}

\thestrophe Nach der Wissenschaft zu streben, sei uns allen ernste Pflicht; \\
nur der Wahrheit lasst uns leben in der Freiheit Himmelslicht! \\
Hohen Zielen aufgeschlossen, gilt's die Tat, den ganzen Mann, \\
\wdh{gehet, Brüder, unverdrossen unserm Volke stets voran!}

\thestrophe Für die Freundschaft, die uns bindet, gebt das Letzte freudig hin! \\
Unser Burschenband verkündet dieses Bundes schönsten Sinn: \\
Uns als Brüder zu bewähren, jeder treu zum Bunde hält.\\
\wdh{Dir will immer ich gehören, heil CV, du meine Welt!}
 %"auf_des_glaubens_N.lytex" wird eingebunden
%Dies ist die Vorlage für die Version ohne Noten

\subsection*{Einer Farbe, einem Glauben}
\index{Einer Farbe, einem Glauben}%
\index{MKV Hymne}%
%
\nopagebreak
\weise{Strömt herbei ihr Völkerscharen}
%
\nopagebreak
\thestrophe Einer Farbe, einem Glauben, einer Sitte zugetan, \\
häng' ich wie die frommen Tauben meiner lieben Heimat an. \\
Wo ich lebe, will ich sterben; wo ich sterbe, ruht sich's gut; \\
\wdh{und die Kinder, die mir erben, erben auch mein Herz, mein Blut.}

\thestrophe Süße Heimat, schöne Erde, gutes Land, das mich erhält, \\
o du teure, liebe, werte, o du kleine heit're Welt! \\
Immer will ich dir gehören, immer mit und bei dir sein! \\
\wdh{Fremdlinge und Söldner schwören, dir genügt mein Wort allein.}

\thestrophe Meinem Glauben, meiner Sitte, meinem Vaterlande treu, \\
kenn' ich weder Wunsch noch Bitte, frage nicht, wo's besser sei. \\
Mögen and're wünschen, suchen, mir sind über Gut und Geld \\
\wdh{meine Eichen, meine Buchen, MKV, du meine Welt!}

%Dies ist die Vorlage für die Version ohne Noten

\subsection*{In dem Städtchen nah am Strande}
\index{In dem Städtchen nah am Strande}%
\index{Korneuburg}%
%
\nopagebreak
\weise{Strömt herbei, ihr Völkerscharen}
%
\nopagebreak
\thestrophe In dem Städtchen nah' am Strande, oft von Feinden hart bedroht, \\
Türken und Hussitenbande unsren Vätern brachten Not. \\
In dem stillen Uferhaine mancher seine Liebste fand; \\
\wdh{leicht erkennst du, was ich meine: 's ist Korneuburg, Heimatland!}

\thestrophe Nach der Kreuzenstein dort oben, schönster Burg im weiten Land, \\
junge Burschen sind gezogen buntbemützt, den Stock zur Hand. \\
Und im Tal tief unt' sie schauen Strom und Städtchen, stolzgeschwellt, \\ 
\wdh{über Fluren, Wälder, Auen: lieb Korneuburg, meine Welt!}

\thestrophe Wenn die Jugend einst vergangen, bitt'res Leid dein Herz bedrückt, \\
ziehe, wo du einst gehangen froh im Kreis, der Welt entrückt, \\
wo man Freud' und Sorgen teilte, wo der Bisamberger Wein \\
\wdh{selbst die größten Schmerzen heilte: in Korneuburg ziehe ein.}

\mychapter{Landeshymnen}

%Dies ist die Vorlage für die Version ohne Noten

\subsection*{Dort, wo Tirol an Salzburg grenzt}
\index{Dort, wo Tirol an Salzburg grenzt}%
\index{Kärntner Landeshymne}%

\thestrophe Dort, wo Tirol an Salzburg grenzt, des Glockners Eisgefilde glänzt, \\
wo aus dem Kranz, der es umschließt, der Leiter reine Quelle fließt, \\
\wdh{laut tosend, längs der Berge Rand, beginnt mein teures Heimatland.}

\thestrophe Wo durch der Matten herrlich Grün des Draustroms rasche Fluten ziehn; \\
vom Eisenhut, wo schneebedeckt sich Nordgau's Alpenkette streckt, \\
\wdh{bis zur Karawanken Felsenwand dehnt sichmein freundlich Heimatland.}

\thestrophe Wo von der Alpenluft umweht, Pomonens schönster Tempel steht, \\
wo durch die Ufer, reich umblüht, der Lavant Welle rauschend zieht, \\
\wdh{im grünen Kleid ein Silberband, schließt sich mein liebes Heimatland.}
%Dies ist die Vorlage für die Version ohne Noten

\subsection*{Du Ländle, meine teure Heimat}
\index{Du Ländle, meine teure Heimat}%
\index{Vorarlberger Landeshymne}%

\thestrophe Du Ländle, meine teure Heimat, ich singe dir zu Ehr'und Preis: \\
Begrüße deine schönen Alpen, wo Blumen blüh'n so edelweiß \\
und golden glühen steile Berge, berauscht vom harz'gen Tannenduft. \\
\wdh{O Vorarlberg, will treu dir bleiben, bis mich der liebe Herrgott ruft!}

\thestrophe Du Ländle, meine teure Heimat, wo längst ein rührig Völklein weilt, \\
wo Vater Rhein, noch jung an Jahren, gar kühn das grüne Tal durcheilt: \\
hier hält man treu zum Vaterlande, und rotweiß weht es durch die Luft. \\
\wdh{O Vorarlberg, will treu dir bleiben, bis mich der liebe Herrgott ruft!}

\thestrophe Du Ländle, meine teure Heimat, wie könnt' ich je vergessen \\
dein, es waren doch die schönsten Jahre beim lieben, guten Mütterlein; \\
drum muss ich wiederkommen, und trennte mich die größte Kluft. \\
\wdh{O Vorarlberg, will treu dir bleiben, bis mich der liebe Herrgott ruft!}
%Dies ist die Vorlage für die Version ohne Noten

\subsection*{Hoamatland}
\index{Hoamatland}%
\index{Oberösterreichische Landeshymne}%

\thestrophe Hoamatland, Hoamatland, die han i so gern, \\
\wdh{wiar a Kinderl sein Muader, a Hünderl sein Herrn.}

\thestrophe Duri 's Tal bin i g'laffn, af'n Höcherl bin i glegn, \\
\wdh{und deinSunn hat mi trückat, wann mi gnetzt hat dein Regn.}

\thestrophe Deine Bam, deine Staudna san groß wom mit mir, \\
\wdh{und sie bliahn sehen und tragn und sagn: Machts a wia mir.}

\thestrophe Am schönem macht's 's Bacherl, laft allweil tal-a, \\
\wdh{aber 's Herz, vo wo's auarinnt, 's Herz, des laßt's da.}

\thestrophe Und i und die Bachquelln san Vetter und Moahm; \\
\wdh{treibt's mi, wo da wöll, umma, mein Herz is dahoam.}

\thestrophe Dahoam is dahoam, wannst net furt muaßt, so bleib, \\
\wdh{denn d'Hoamat is ehnta da zweit Muaderleib.}


\mychapter{Orte}

%Dies ist die Vorlage für die Version ohne Noten

\subsection*{Als ich zog zur Alma Mater}
\index{Als ich zog zur Alma Mater}%
\index{Studienstadt@Köln}%
\nopagebreak
\hfill%
Studienstadt: Köln\\
\weise{Heidelberg, du Jugendbronnen}
\nopagebreak
\thestrophe Als ich zog zur Alma Mater, trieb es mich nach Köln am Rhein. \\
Warnte auch der gute Vater: "Filius, was fällt dir ein! \\
O ich kenne Köln, das Städtchen, schmeckt dort gar zu gut der Wein! \\
\wdh{Und von vielen rhein'schen Mädchen locken dich die Äugelein!}

\thestrophe Doch ich ließ mich nicht berücken, grüßte bald den alten Dom. \\
Und mit wonnigem Entzücken winkte ich dem grünen Strom. \\
Sah in Frühlingsblumen prangen weit und breit des Rheines Flur! \\
\wdh{Und viel frohe Burschen sangen: Gaudeamus igitur!}

\thestrophe Und ich dacht' des Vaters Worte, schritt fürbaß dann durch die Stadt, \\
schaute suchend nach dem Orte, wo dereinst dozieret hat: \\
Albert Magnus sondern Fehle! Hab' ihm manchen Schluck geweiht, \\
\wdh{wenn ich saß mit durst'ger Kehle Winters und zur Sommerzeit!}

\thestrophe Möcht euch nun gern eins singen von so schnell verfloss'rier Zeit, \\
von Colleg und Becherschwingen, von der Maid, die ich gefreit! \\
Doch es möge jeder leise reimen sich die Melodei! \\
\wdh{Denn nach guter alter Weise sind stets Sang und Lehre frei!}

\thestrophe Lasst nun froh die Gläser klingen, schnell verrinnt der Jugend Zeit – \\
und ein kräftig Schmollis bringen Kölner Burschenherrlichkeit.\\
Wenn auch einstens scheiden müssen wir vom Liebchen und vom Rhein: \\
\wdh{Mädchen, lass dich herzlich küssen, unser Gruß gilt stets euch zwei'n.}

%Dies ist die Vorlage für die Version ohne Noten

\subsection*{Wien, mein Wien, gar oft besungen}
\index{Wien, mein Wien, gar oft besungen}%
\index{Studienstadt@Wien}%
%
\nopagebreak
\hfill%
\comment{Studienstadt: Wien}
\weise{Heidelberg, du Jugendbronnen}
%
\nopagebreak
\thestrophe Wien, mein Wien, gar oft besungen, an der blauen Donau Strand, \\
schon im Lied der Nibelungen bist ein Kleinod du genannt. \\
Wien, mein Wien, du Stolz der Ahnen, schönes Wien, demantengleich, \\
\wdh{fröhlich flattern deine Fahnen in dem schönen Österreich!}

\thestrophe Wo dem Nebeldunst enthoben dich der Wienerwald umrauscht,\\
dort die alte Feste droben mit dem Himmel Zwiesprach' tauscht. \\
Drunten aber fest und trutzig, treu bewährt in manchem Sturm, \\
\wdh{grüßt uns aus dem Häusermeere unser alter Stephansturm!}

\thestrophe Prinz Eugen und Kaiser Josef, Mozart, Raimund, Schubert Franz, \\
sind im Herzen uns verwoben wie der alte Wiener Tanz. \\
Alte, liebe Wiener Sitte, alte Wiener Fröhlichkeit, \\
\wdh{Strauß und Lanner, Wiener Walzer, leben fort in Ewigkeit!}

\thestrophe Und da draußen vor den Toren grüßt ein Städtchen traut im Land, \\
wo das Liebste uns geboren, das auf Erden uns bekannt. \\
Wo das blau-weiß-gold'ne Banner flatterte zuerst im Wind, \\
\wdh{in Korneuburgs engen Gassen fühlt man als der Heimat Kind!}

\thestrophe So bleibt ihr uns stets verwoben: Stadt und du, mein Städtchen traut, \\
euch zum Preise sei erhoben, was Gambrinus uns gebraut. \\
Ob uns mahnen Kirchenglocken, ob uns ruft Kommersgesang, \\
\wdh{ob zum Tanz die Geigen locken: stets ist's unsrer Heimat Klang!}


\mychapter{Officium}

%Dies ist die Vorlage für die Version ohne Noten

\subsection*{Gaudeamus igitur}
\index{Gaudeamus igitur}%
\index{Erstes Allgemeines}%

\thestrophe \wdh{Gaudeamus igitur, iuvenes dum sumus,} \\
post iucundam iuventutem, post molestam senectutem \\
\wdh{nos habebit humus.}

\thestrophe \wdh{Ubi sunt, qui ante nos in mundo fuere?} \\
Vadite ad superos, transite ad inferos, \\
\wdh{ubi iam fuere.}

\thestrophe \wdh{Vita nostra brevis est, brevi finietur.} \\
Venit mors velociter, rapit nos atrociter, \\
\wdh{nemini parcetur.}

\thestrophe \wdh{Vivat academia, vivant professores,} \\
vivat membrum quodlibet, vivant membra quaelibet \\
\wdh{semper sint in flore!}

\thestrophe \wdh{Vivant omnes virgines, faciles, formosae,} \\
vivant et mulieres, tenerae, amabiles, \\
\wdh{bonae, laboriosae!}

\thestrophe \wdh{Vivat et res publica et qui illam regit!} \\
Vivat nostra civitas, maecenatum caritas, \\
\wdh{quae nos hie protegit!}

\thestrophe \wdh{Pereat tristitia, pereant osores,} \\
pereat diabolus, quivis antiburschius \\
\wdh{atque irrisores!}
%Dies ist die Vorlage für die Version ohne Noten

\subsection*{Alles schweige! Jeder neige}
\index{Alles schweige! Jeder neige}%

\weise{Landesvaterzeremonie - Vor der Burschung}

\thestrophe Alles schweige! Jeder neige ernsten Tönen nun sein Ohr! \\
\wdh{Hört, ich sing das Lied der Lieder! Hört es, meine Bundesbrüder! \\
\wdh{Hall es} wider, froher Chor!}

\thestrophe Öst'rreichs Söhne, laut ertöne euer Vaterlandsgesang! \\
\wdh{Vaterland! Du Land des Ruhmes, weih' zu deines Heiligtumes \\
\wdh{Hütern} uns und unser Schwert!}

\thestrophe Hab und Leben dir zu geben, sind wir allesamt bereit, \\
\wdh{sterben gern zu jeder Stunde, achten nicht der Todeswunde, \\
\wdh{wenn das} Vaterland gebeut.}

\thestrophe Wer's nicht fühlet, selbst nicht zielet stets nach treuer Männer \\
Wert, \wdh{soll nicht unsern Bund entehren, nicht bei diesem Schläger schwören, \\
\wdh{nicht ent-} weih'n das starke Schwert.}

\thestrophe Lied der Lieder, hall' es wider: groß und stark sei unser Mut! \\
\wdh{Seht hier den geweihten Degen, tut, wie brave Burschen pflegen, \\
\wdh{und durch-} bohrt den freien Hut!}

\thestrophe Seht ihn blinken in der Linken diesen Schläger, nie entweiht! \\
\wdh{Ich durchbohr' den Hut und schwöre, halten will ich stets auf Ehre; \\ 
\wdh{stets ein} braver Bursche sein.}

\thestrophe Nimm den Becher, wack'rer Zecher, vaterländ'schen Trankes voll; \\ 
\wdh{nimm den Schläger in die Linke, bohr ihn durch den Hut und trinke \\
\wdh{auf des} Vaterlandes Wohl!}

\thestrophe Seht ihn blinken in der Linken diesen Schläger, nie entweiht! \\
\wdh{Ich durchbohr' den Hut und schwöre, halten will ich stets auf Ehre; \\
\wdh{stets ein} braver Bursche sein.}

\thestrophe Komm, du blanker Weihedegen, freier Männer freie Wehr! \\
bringt ihn festlich mir entgegen von durchbohrten Hüten schwer.

\thestrophe Laßt uns festlich ihn entlasten; jeder Scheitel sei bedeckt, \\
und dann laßt ihn unbefleckt bis zu nächsten Feier rasten!

\thestrophe Auf, ihr Festgenossen, achtet uns're Sitte heilig schön! \\
Ganz mit Herz und Seele trachtet, stets als Männer zu besteh'n. \\
Froh zum Fest, ihr trauten Brüder; jeder sei der Väter wert! \\
Keiner taste je an's Schwert, der nicht edel ist und bieder!

\thestrophe So nimm ihn hin dein Haupt will ich bedecken und drauf den \\
Schläger strecken: es leb' auch dieser Bruder hoch! \\
Ein Hundsfott, wer ihn schimpfen sollt! So lange wir ihn kennen, \\
woll'n wir ihn Bruder nennen: es leb' auch dieser Bruder hoch!

\thestrophe Ruhe von der Burschenfeier, blanker Weihedegen, nun! \\
Jeder trachte, wackrer Freier um das Vaterland zu sein! \\
Jedem Heil, der sich bemühe, ganz zu sein der Väter wert; \\
keiner taste je ans Schwert, der nicht edel ist und bieder.

\pagebreak
\printindex

\end{document}
